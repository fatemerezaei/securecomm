\section{Conclusions}

The reliable access to \bc and similar cryptocurrencies is of crucial importance due to
their consumers and the related industry. 
In this paper, we investigated the resilience of \bc to blocking by a powerful network entity such as an ISP or a government.
By characterizing \bc's communication patterns, we designed 
various classifiers that could distinguish (and therefore block) \bc traffic 
even if it is tunneled over an encrypted channel like VPN or Tor, 
and even when it is mixed with background traffic. Through extensive experiments on network traffic, we demonstrated that our classifiers could reliably identify \bc traffic despite using obfuscation protocols like Tor pluggable transports that modify traffic patterns. 

We learn from our experiments that it is
extremely hard to hide Bitcoin traffic using standard obfuscation mechanism due to specific protocol messages with unique sizes and frequencies. In order to disguise such patterns, an obfuscating protocol needs to apply significant cover traffic or employ large perturbations, which is undesirable for typical clients. 
We suggest that future work should look into designing obfuscation protocols that are tailored to \bc (and similar cryptocurrencies) in a way to optimize resilience to detection and resource efficiency. 
 


